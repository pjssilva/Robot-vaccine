\documentclass{article}

\usepackage{amsmath}
\usepackage{hyperref}

\newcommand{\I}{{\mathcal{I}}}

\usepackage{tikz}
\usetikzlibrary{shapes,arrows,positioning,calc}
\title{Simulation 1}

\begin{document}

\section{Parameters}

\begin{itemize}
    \item $T_{inc}$: mean incubation time in days. I am using 5.2 as in the SIREV paper.

    \item $T_{inf}$: mean infectious time in days. I am using 2.9 as in the SIREV paper.

    \item $R_0$: 1.8 to model a ``new normal'' as in the SIREV paper.

    \item Initial SEIR state for each population: obtained roundig (down) the
    values obatining from SP: S0=0.685751 E0=0.009905 I0=0.005572 R0=0.298772. This was obtained using a subnotification factor of 7.6 to match the recovered proportion described in \href{https://www1.folha.uol.com.br/equilibrioesaude/2021/02/em-sp-um-terco-dos-adultos-tiveram-covid-taxa-sobe-a-quase-40-entre-negros.shtml}{a recent sorolo inquiry published in Folha}.
        
    {\color{red} I am using then $S0=68.5\%, E0=1.0\%, I0=0.6\%, R0=29.9\%$. I
    will assume that the subpopulations will conform to this distribution
    according to their size.} 
    
    \item The time window to choose a different social distancing profile
    ($r_t$). I am using 14 days as before. 

    \item The age groups: {\color{red} The age groups will be $[0, 19],\ [20, 49], [50, 64],  [65, \infty)$.} 
    
    \item Relative size of each subpopulation. Using the census from 2010 in SP, I get to: 30\%, 48\%, 14\%, 8\%. 

    \item Mean ICU time: I am using 7 days in the Robot test paper.
    
    \item Number of ICUs available (Claudia): population = 44,639,899, Available
    ICU 7812. Ratio of Available ICU = 0,0175\% (or 17.5 for 100K habitants).
    
    \item Mean ICU demand: SP data suggest mean ICU necessity of
    0.00956607 (the ratio $ICU_t$ in SiRev). The time series would be this
    constant value plus a white noise with standard deviation 0.00399146, both
    scaled. Since I am using "rounded" versions of SP data in this test I will
    employ ratio 1\% and the standard deviation of the white noise equal
    0.004. {\color{red} But we still need to come up with a way to derive from
    these numbers the ICU necessity of each population.}

    \item Factor to correct ICU demand: $[0, 19]:\ 0.06$, $[20, 49]:\ 0.58$,
    $[50, 64]:\ 2.06$ e $[65, 200]:\ 5.16$.
    
    \item The number of days needed for one and two doses to make effect: 14
    days for both (Claudio, Tiago). This is the amount of time after the vaccine
    that allow the transition for the lower states.
    
    \item Minimum number of days before the second dose (I am assuming that it
    is at least $T_{inc} / 2$ longer than the number of days needed for the
    vaccine to make effect): 28 (Claudio, Tiago).

    \item Maximum number of days between the two doses: 3 months = 90 days.
    
    \item Number of vaccines doses that can be deployed each day (as a
    proportion of the population size): 0.1\% for 30 days (up to March), 0.5\%
    from 31 to 150 (March to June)  1.5\% from there on (Claudio, Tiago).
    
    \item How much the number of doses attenuate the need for ICU beds. If used
    I will go with 50\% for one dose and 70\% for two doses. (Claudio, Tiago). 

    \item $a_p$: How much one and two doses attenuate transmission: again
    50\% and 70\% if used. (Claudio, Tiago)

    \item $b_p$: A factor to multiply the overall $R0$ of each subpopulation to
    say whether subpopulation $p$ is more or less susceptible than the overall
    population: {\color{red} I will use $1.0$ for $[0, 19]$, $1.3$ for $[20,
    49]$, $1.0$ for $[50, 64]$ and $1.0$ for $[65, \infty)$.} (Claudio, Tiago).

    \item $C$: the contact matrix (computed from the contact matrix given by Thiago, $C_{i, j}$ among the contacts of $i$ the proportion that it is made
    with people from $j$):

    \begin{verbatim}
                0    20   50   65
                19   49   64   200
        0  19  0.57 0.27 0.10 0.06
        20 49  0.20 0.59 0.15 0.06
        50 64  0.15 0.46 0.27 0.12
        65 200 0.18 0.24 0.18 0.39
    \end{verbatim}
    
    \item Objective: minimize the number where some kind of social distancing is
    needed.
\end{itemize}

\end{document}